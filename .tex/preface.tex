\documentclass[a4paper,12pt]{article}
\usepackage{mystyle}

\begin{document}
\section*{本稿について}
\addcontentsline{toc}{section}{本稿について}
本稿は, 2014年頃に有志メンバーによって開催された勉強会における発表資料を整理したものである. 本稿の目標は, 局所二乗可積分な Martingale を Integrator とする Stochastic Integral の構成法 (定理 \ref{construction of second SI}) を紹介し, Stochastic Integral と Lebesgue-Stieltjes 積分等の, その他の積分概念との関係を手早く示すこと (定理 \ref{relation with IS integral}, 命題 \ref{relation with LS integral}) に限定している. 予備知識として, 標準的な測度論と入門的な連続時間 Martingale についての知識を想定している. 後者については, たとえば, \cite{Karatzas-Shreve:bmsc} の Chapter 1 や, \cite{Rogers-Williams:dmm1} の Section 5 を挙げておく. 

なお, 本稿で述べる内容は, より広い枠組みである Semimartingale を用いた Stochastic Integral で扱うことができる. より一般的な議論は \cite{Rogers-Williams:dmm2}\cite{Protter:si-and-de}\cite{Medvegyev:sit}\cite{Karatzas-Shreve:bmsc}などを参照されたい. 本稿は, 主に \cite{Medvegyev:sit}\cite{KunitaWatanabe} を参考した.
\end{document}